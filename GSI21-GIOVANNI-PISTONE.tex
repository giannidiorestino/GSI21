% This is samplepaper.tex, a sample chapter demonstrating the
% LLNCS macro package for Springer Computer Science proceedings;
% Version 2.20 of 2017/10/04
%
\documentclass[runningheads]{llncs}
%
\usepackage[hyphens]{url}
\usepackage[pdftex]{graphicx}
\usepackage{amsmath,amssymb}
%\usepackage{accents}
%\usepackage{amsthm}
\usepackage{array}
\usepackage{bm}
\usepackage{color}
\usepackage{etex}
\usepackage{extramacros}
%\usepackage{fullpage}
\usepackage{graphicx}
\usepackage{hyperref}
\usepackage{mathtools}
\usepackage{stackrel}
\usepackage{cleveref}

%\usepackage{graphicx}
% Used for displaying a sample figure. If possible, figure files should
% be included in EPS format.
%
% If you use the hyperref package, please uncomment the following line
% to display URLs in blue roman font according to Springer's eBook style:
\renewcommand\UrlFont{\color{blue}\rmfamily}


\begin{document}
%
\title{Statistical bundle of the transport model\thanks{The author is
    supported by de Castro Statistics, Collegio Carlo Alberto. He is a member of INDAM-GNAMPA.}}
%
%\titlerunning{Abbreviated paper title}
% If the paper title is too long for the running head, you can set
% an abbreviated paper title here
%
\author{Giovanni Pistone\inst{1}\orcidID{0000-0003-2841-788X}}
%
\authorrunning{G. Pistone}
% First names are abbreviated in the running head.
% If there are more than two authors, 'et al.' is used.
%
\institute{De Castro Statistics, Collegio Carlo Alberto, Piazza
  Arbarello 8, 10122 Torino, Italy \\
  \email{giovanni.pistone@carloalberto.org}\\
  \url{https://www.giannidi orestino.it}}
%
\maketitle              % typeset the header of the contribution
%
\begin{abstract}
  We discuss the statistical bundle of the Kantorovich transport
  model. We assume a finite sample space. We discuss the relation of
  the splitting of the general statistical bundle both with the
  gradient flow of the Kantorovich cost and with the statistics of
  exponential additive models. This preliminary version is submitted
  to GSI21 as an extended abstract. For this reason, no primary references are
  given.  \keywords{Statistical bundle \and Kantorovich transport
    model \and Gradient flow \and Exponential additive model}
\end{abstract}

Consider a product finite sample space
$\Omega=\Omega_1 \times \Omega_2$ and the marginal mappings
$X \colon \Omega \to \Omega_1$, $Y \colon \Omega \to \Omega_2$. If
$\gamma$ is a probability function on $\Omega$, we denote by
$\gamma_1$ and $\gamma_2$ the image probability functions on
$\Omega_1$ and $\Omega_2$, respectively. We denote by $L(\gamma)$,
$L(\gamma_1)$ and $L(\gamma_2)$ the Euclidean spaces of real random
variables and by $L_0(\gamma)$, $L_0(\gamma_1)$ and $L_0(\gamma_2)$
the spaces of random variables which are centered for the given
probability function.

We use the non-parametric version of the Information Geometry of the
finite sample space described in the tutorial \cite{pistone:2020-NPCS}
and applied in \cite{pistone|rogantin:2015.1502.06718}
\cite{pistone:2018lagrange},
\cite{chirco|malago|pistone:2020-arXiv:2009.09431}. In this setup,
$\Delta^\circ(\Omega)$ is the interior of the probability simplex and
$S = S\Delta^\circ(\Omega)$ is the \emph{statistical bundle}, that is,
the set of all couples $(\gamma,u)$ with
$\gamma \in \Delta^\circ(\Omega)$ and $u$ a random variable such that
$\expectat \gamma u=0$. For each smooth curve
$t \mapsto \gamma(t) \in \Delta^\circ(\Omega)$ the \emph{velocity}
$\velocity \gamma(t) = \derivby t \log \gamma(t)$ is characterised by
\begin{equation}
  \label{eq:1}
  \derivby t \expectat {\gamma(t)} u = \scalarat {\gamma(t)} {u - \expectat {\gamma(t)} u} {\velocity
  \gamma(t)} \ .
\end{equation}
The mapping $\gamma \mapsto u - \expectat \gamma u$ is the
\emph{gradient} of $\gamma \mapsto U(\gamma) = \expectat \gamma u$

On a sub-manifold, each fiber $S_\gamma$ of the
statistical bundle splits to define a proper sub-statistical bundle. 
We are interested in the sub-manifold of tranport plans. Let be given
$\gamma_1 \in \Delta^\circ(\Omega_1)$ and
$\gamma_2 \in \Delta^\circ(\Omega_2)$. The \emph{transport model}
with margins $\gamma_1$ and $\gamma_2$ is the statistical model
%
  \begin{equation*}
    \Gamma(\gamma_1,\gamma_2) = \setof{\gamma \in \Delta(\Omega)}{X_\# \gamma = \gamma_1, Y_\# \gamma = \gamma_2} \ .
  \end{equation*}
%
  Without restriction of generality, we assume $\gamma_1$ and
  $\gamma_2$ to be strictly positive. The \emph{open transport model}
  is
%
  \begin{equation*}
    \Gamma^\circ(\gamma_1,\gamma_2) = \setof{\gamma \in \Delta^\circ(\Omega)}{X_\# \gamma = \gamma_1, Y_\# \gamma = \gamma_2} \ .
  \end{equation*}
  %
  Marginalization acts on velocities as a conditional expectation. If
  $t \mapsto \gamma(t)$ is a smooth curve in the open transport model,
  then \cref{eq:1} with $u = f(X)$ gives
  \begin{equation*}
  0 = \derivby t \expectat {\gamma_1} f =  \derivby t \expectat
  {\gamma(t)} {f(X)} = \scalarat {\gamma(t)}  {f(X)} {\velocity \gamma(t)}
  = \scalarat {\gamma_1}  f {{\velocity \gamma(t)}_1} \ , 
  \end{equation*}
with ${\velocity \gamma(t)}_1(X) = \condexpectat {\gamma(t)} {\velocity
  \gamma(t)} X$. It follows that
\begin{equation*}
  \condexpectat {\gamma(t)} {\velocity
  \gamma(t)} X = 0 \quad \text{and} \quad \condexpectat {\gamma(t)} {\velocity
  \gamma(t)} Y = 0 \ .
\end{equation*}

Let us discuss more in detail the relevant splitting in the language
of Analysis of Variance(ANOVA). Let $\gamma$ be
a probability function on $\Omega$. The linear subspaces of the space
of random variables $L(\Omega)$ which describe, respectively, the
\emph{$\gamma$-grand-mean}, the two \emph{$\gamma$-marginal effects}, and
the \emph{$\gamma$-interactions}, are
%
\begin{equation}\label{eq:ANOVA-spaces}
\begin{aligned}
  B_0(\gamma) &\sim \reals, \\
  B_1(\gamma) &= \setof{f\circ X}{f \in L_0(\gamma_1)},\\
  B_2(\gamma) &= \setof{f\circ Y}{f \in L_0(\gamma_2)},\\
  B_{12}(\gamma) &= (B_0(\gamma) + B_1(\gamma) + B_2(\gamma))^\perp,
\end{aligned}
\end{equation}
%
where orthogonality is computed in the $\gamma$ weight, that is in the scalar product
%%
\begin{equation*}
  (u,v) \mapsto \scalarat \gamma u v = \sum_{\omega\in\Omega}u(\omega)v(\omega)
  \gamma(\omega)\end{equation*}

Each element of the space $B_0(\gamma) + B_1(\gamma) + B_2(\gamma)$
has the form $u = u_0 + f_1(X) + f_2(Y)$, where
$u_0 = \expectat \gamma u$ and $f_1,f_2$ are uniquely defined. More
precisely, we have the following.

\paragraph{$\bm\models$} \textit{For each $\gamma \in \Delta(\Omega)$ there exist a splitting
%
\begin{equation*}
%
  L(\gamma)) = \reals \oplus \left(B_1(\gamma) + B_2(\gamma)\right) \oplus B_{12}(\gamma) \ . 
\end{equation*}
%
Namely, the splitting of each $u \in L(\gamma)$ is
%
\begin{equation}
 \label{eq:ANOVA}
 u = u_0 + u_1 + u_2 + u_{12} \ ,
\end{equation}
%
where $u_0=\expectat \gamma u$ and $u_1+u_2$ is the
$\gamma$-orthogonal projection of $u$ unto
$B_1(\gamma) + B_2(\gamma)$.}

Each couple of spaces in \cref{eq:ANOVA-spaces} has trivial
intersection $\set{0}$, hence the splitting of $u \in L(\Omega)$ into
is unique. As $\expectat \gamma u = u_0$ and $B_0(\gamma)$ is
orthogonal to $B_1(\gamma)+B_2(\gamma)$, then $u_{12}$ is the
orthogonal projection of $u - \expectat \gamma u$ unto
$(B_1(\gamma)+B_2(\gamma))^\perp$, while $f_1(\gamma)+f_2(\gamma)$ is
the orthogonal projection of $u$ onto $B_1(\gamma)+B_2(\gamma)$.

\paragraph{$\bm\models$}\emph{The terms $u_1$ and $u_2$ in Eq. \cref{eq:ANOVA} are characterized by the equations
%
\begin{equation}
\label{eq:projetion}
  \begin{aligned}
    \condexpectat \gamma {(u - u_0)} X u_1 + \condexpectat \gamma {(u-u_0)u_2} X &= 0 \ , \\
    \condexpectat \gamma {(u - u_0)u_1} Y + \condexpectat \gamma {(u-u_0)} Y u_2 &= 0 \ .
  \end{aligned}
%
\end{equation}}

By definition, $(u - u_0) - (u_1 + u_2)$ is $\gamma$-orthogonal to each $g_1+g_2 \in B_1(\gamma)\oplus B_2(\gamma)$ i.e., to each $g_1 \in B_1(\gamma)$ and each $g_2 \in B_2(\gamma)$. The orthogonal projection in such a case is the conditional expectation,
%
  \begin{align*}
    0 &= \condexpectat \gamma {(u-u_0) - (u_1+u_2)}{X}  \ , \\
   0 &= \condexpectat \gamma {(u-u_0) - (u_1+u_2)}{Y} \ .
  \end{align*}
  %

Expressing the conditional expectations with respect to the uniform distribution,
%
  \begin{align*}
    0 &= \condexpectat 0 {\gamma[(u-u_0) - (u_1+u_2)]}{X} \ , \\
   0 &= \condexpectat 0 {\gamma[(u-u_0) - (u_1+u_2)]}{Y} \ ,
  \end{align*}
  % 
that is
%
  \begin{align*}
  \condexpectat 0 {\gamma(u-u_0)} X =  \condexpectat 0 {\gamma} X u_1 + \condexpectat 0 {\gamma u_2} X \ , \\
    \condexpectat 0 {\gamma(u-u_0)} Y =  \condexpectat 0 {\gamma u_1} X  + \condexpectat 0 {\gamma} Y u_2\ .
  \end{align*}
  % 

The ANOVA decomposition proves the existence of the splitting and suggests a matric computation. Assume zero mean, $u_0=0$. The following $(n_1+n_2)\times(n_1+n_2)$ linear system holds:
%
\begin{equation*}
\begin{cases}
  \sum_{y \in \Omega_2} \gamma(x,y)u(x,y) &= \gamma_1(x) u_1(x) + \sum_{y \in \Omega_2} \gamma(x,y) u_2(y) , \quad x \in \Omega_1 \\
  \sum_{x \in \Omega_1} \gamma(x,y)u(x,y) &= \sum_{x \in \Omega_1} \gamma(x,y) u_1(x)  + \gamma_2(y)u_2(y) , \quad y \in \Omega_2
\end{cases}
\end{equation*}
%

Equivalently, using the conditional probabilities $\gamma_{2|1}(y|x) \gamma_1(x) = \gamma_{1|2}(x|y) \gamma_2(y) = \gamma(x,y)$, $x \in \Omega_1$ and $y \in \Omega_2$,
%
\begin{equation*}
\begin{cases}
  \sum_{y \in \Omega_2} \gamma_{2|1}(y|x)u(x,y) &= u_1(x) + \sum_{y \in \Omega_2} \gamma_{2|1}(y|x) u_2(y) , \quad x \in \Omega_1 \\
  \sum_{x \in \Omega_1} \gamma_{1|2}(x|y)u(x,y) &= \sum_{x \in \Omega_1} \gamma_{1|2}(x|y) u_1(x)  + u_2(y) , \quad y \in \Omega_2
\end{cases}
\end{equation*}
%

We write the previous system in matrix form with blocks
\begin{gather*}
  \Gamma_{2|1} = [\gamma_{2|1}(y|x)]_{x \in \Omega_1, y \in \Omega_2} \in \reals^{n_1 \times n_2} \\
  \Gamma_{1|2} = [\gamma_{1|2}(x|y)]_{x \in \Omega_1, y \in \Omega_2} \in \reals^{n_1 \times n_2} \\
  \bm u_1 = (u_1(x) | x \in \Omega_1) \\
  \bm u_2 = (u_2(x) | x \in \Omega_2) \\
  \bm u_{2|1} = (\sum_{y \in \Omega_2} u(x,y)\gamma_{2|1}(y|x) | x \in \Omega_1) \\
  \bm u_{1|2} = (\sum_{x\in\Omega_1} u(x,y)\gamma_{1|2}(x|y) | x \in \Omega_2) \ ,
\end{gather*}
as 
\begin{equation}
\label{eq:block}
  \begin{bmatrix}
    I_{n_1} & \Gamma_{2|1} \\ \Gamma_{1|2}^T & I_{n_2}
  \end{bmatrix}
  \begin{bmatrix}
    \bm u_1 \\ \bm u_2
  \end{bmatrix}
=
\begin{bmatrix}
  \bm u_{2|1} \\ \bm u_{1|2}
\end{bmatrix}
\end{equation}

The block matrix is not globally invertible because the kernel contains the vector $\bm 1 _{n_1} \oplus \bm 1_{n_2}$, hence we look for a generalised inverse of the block matrix  which  exists uniquely because of the interpretation as the orthogonal projection of $f \in L(\Omega) \ominus L_0(\gamma)$ onto $B_1(\gamma)\oplus B_2(\gamma)$. Generalised inverses of the Shur complements $(I_{n_1}-\Gamma_{2|1}\Gamma_{1|2}^T)$ and $(I_{n_2}-\Gamma_{1|2}^T\Gamma_{2|1})$, if available, suggests the following generalised block inversion formula:
%
\begin{equation}
\label{eq:blocksolve}
\begin{bmatrix}
    I_{n_1} & \Gamma_{2|1} \\ \Gamma_{1|2}^T & I_{n_2}
  \end{bmatrix} ^+ =
  \begin{bmatrix}
    (I_{n_1}-\Gamma_{2|1}\Gamma_{1|2}^T)^{+} & - \Gamma_{2|1} (I_{n_2}-\Gamma_{1|2}^T\Gamma_{2|1})^{+} \\
 - \Gamma_{1|2}^T (I_{n_1}-\Gamma_{2|1}\Gamma_{1|2}^T)^{+} & (I_{n_2}-\Gamma_{1|2}^T\Gamma_{2|1})^{+} 
\end{bmatrix} \ 
\end{equation}

\paragraph{$\bf\models$} \emph{The matrices $P = \Gamma_{2|1}\Gamma_{1|2}^T$ and $Q = \Gamma_{1|2}^T\Gamma_{2|1}$ are Markov, with invariant probability $\gamma_1$ and $\gamma_2$ respectively, and strictly positive, hence ergodic.}

The element $P_{x_1x_2}$, $x_1,x_2 \in \Omega_1$, of $P = \Gamma_{2|1}\Gamma_{1|2}^T \in \reals^{\Omega_1 \times \Omega_2}$ is
%
\begin{equation*}
  P_{x_1x_2} = \sum_{y \in \Omega_2} \gamma_{2|1}(y|x_1) \gamma_{1|2}(x_2|y) \ ,
\end{equation*}
%
so that $P$ is a Markov matrix with strictly positive entries and stationary probability $\gamma_1$. because of the ergodic theorem, $P^n \to P^\infty = \bgamma_1^T \bm 1$, $n \to \infty$, so that 
%
\begin{equation*}
\lim_{n\to\infty} \left(\sum_{k=0}^n P^k\right)(I-P) = \lim_{n\to\infty} \left(I - P^{n+1}\right) = I - P^\infty \ , 
\end{equation*}
%
so that for each $\bm f_1 \in L(\Omega_1)$ it holds
%
\begin{equation*}
  \lim_{n\to\infty} \left(\sum_{k=0}^n P^k\right)(I-P) \bm f_1 = \bm f_1 - \expectat {\gamma_1} {\bm f_1} \ ,
\end{equation*}
%
where the last term is the orthogonal projection $\Pi_{\gamma_1}$ of $\bm f_1$ onto $L_0(\gamma_1)$. \qed

\paragraph{$\bf\models$} \emph{The generalised inverses $(I-P)^+$ and $(I-Q)^+$ are defined and 
%
  \begin{equation*}
  \begin{bmatrix}
    \bm f_1 \\ \bm f_2
  \end{bmatrix}
=
    \begin{bmatrix}
      (I - P)^+ & - \Gamma_{2|1}(I - Q)^+ \\
- \Gamma_{1|2}(I-P)^+ & (I-Q)^+
    \end{bmatrix}
    \begin{bmatrix}
      \bm f_{2|1} \\ \bm f_{1|2}
    \end{bmatrix}
  \end{equation*}
%
solves \cref{eq:block}.}

  Let us write
%
\begin{equation*}
  (I_{n_1}-\Gamma_{2|1}\Gamma_{1|2}^T)^{+} = \lim_{n\to\infty} \left(\sum_{k=0}^n (\Gamma_{2|1}\Gamma_{1|2}^T)^k\right) \Pi_{\gamma_1}  
\end{equation*}
%
so that
%
\begin{equation*}
  (I_{n_1}-\Gamma_{2|1}\Gamma_{1|2}^T)^{+}(I_{n_1}-\Gamma_{2|1}\Gamma_{1|2}^T) = \Pi_{\gamma_1} \ ,
\end{equation*}
%
because $\Pi_{gamma_1} (I_{n_1}-\Gamma_{2|1}\Gamma_{1|2}^T) = (I_{n_1}-\Gamma_{2|1}\Gamma_{1|2}^T)$.

Similarly, for $Q = \Gamma_{2|1}^T\Gamma_{1|2}$, we have
%
\begin{equation*}
  (I_{n_1}-\Gamma_{2|1}^T\Gamma_{1|2})^{+}(I_{n_1}-\Gamma_{2|1}^T\Gamma_{1|2}) = \Pi_{\gamma_2} \ ,
\end{equation*}
%
It follows that \cref{eq:block} is solved with \cref{eq:blocksolve}.\qed

We now apply the ANOVA decomposition to the study of the open
transport model geometry as a sub-bundle of the statistical bundle
$S\Delta^\circ(\Omega)$. Recall that the transport model
$\Gamma(\gamma_1,\gamma_2)$ is a closed convex subset of
$\Delta(\Omega)$. We assume the marginal probability function
$\gamma_1$ and $\gamma_2$ to be strictly positive. The algebraic
interior of the transport model is the open transport model
$\Gamma^\circ(\gamma_1,\gamma_2) \subset \Delta^\circ(\Omega)$. The
interior is never empty as it contains contains
$\gamma_1 \otimes \gamma_2$. The sub-bundle is characterized by the
following two statements.

\paragraph{$\bm\models$} \emph{Let $t \mapsto \gamma(t) \in \Gamma^\circ(\gamma_1,\gamma_2)$ be a smooth curve with $\gamma(0)=\gamma$. Let $B_\gamma = S_\gamma \Delta^\circ(\Omega)$ be the fiber at $\gamma$ of the statistical bundle. Write the ANOVA decomposition as
  \begin{equation*}
    \Bspaceat \gamma = (B_1(\gamma) + B_2(\gamma)) \oplus B_{12}(\gamma) \ .
  \end{equation*}
  Then the velocity at $\gamma$ belongs to the interactions, $\velocity \gamma(0) \in B_{12}(\gamma)$.}


We already know that $\expectat {\gamma} {\velocity\gamma(0)} = 0$. For each $f \in L_0(\gamma_1)$, we have $f \circ X \in B_1(\gamma)$, so that 
%
    \begin{equation*}
    \scalarat \gamma {f \circ X}{\velocity\gamma(0)} = \left. \derivby t \expectat {\gamma(t)} {f \circ X} \right|_{t=0} = \left. \derivby t \expectat {X_{\#}\gamma(t)} {f} \right|_{t=0} = \left. \derivby t \expectat {\gamma_1} {f} \right|_{t=0} = 0 \ .
    \end{equation*}
%
Same argument in the other margin. \qed

\paragraph{$\bm\models$} \emph{Given any interaction $v \in B_{12}(\gamma)$, the curve $t \mapsto \gamma(t) = (1+tv)\gamma$ stays in $\Gamma^\circ(\gamma_1,\gamma_2)$ for $t$ in a neighborhood of 0 and $v = \velocity\gamma(0)$.} 

If $V \in B_{12}(\gamma)$, there exist an open interval I around 0 such that $I \ni t \mapsto \gamma(t) = (1+tV)\gamma$ is a regular curve in $\Gamma^\circ(\mu_1,\mu_2) \subset \Delta^\circ(\Omega)$. It is indeed a curve in $\Delta^\circ(\Omega)$ because $\sum_{x,y} \gamma(x,y)(1+tv(x,y)) = \expectat \gamma {1+tv} = 1$ and because $\gamma(x,y)(1+tV(x,y)) > 0$ for all $x,y$, provided $t \in ]- (\max v)^{-1}, - (\min v)^{-1}[$.  The velocity exists,
%
\    \begin{equation*}
     \velocity\gamma(0) = \left.  \derivby t \logof{(1+tv)\gamma} \right|_{t=0} = \left. \frac v {1+tv} \right|_{t=0} = v \ .
    \end{equation*}
%
Finally, let us compute the margins of $(x,y) \mapsto \gamma(x,y;t) = (1+tv(x,y))\gamma(x,y)$. For all $x \in \Omega_1$,
%
\begin{multline*}
  \sum_y (1+tv(x,y))\gamma(x,y) = X_\# \gamma(x) + t \sum_y v(x,y)\gamma(x,y) = \\ \mu_1(x) + t \expectat \gamma {e_x\circ X V} = \mu_1(x) \ ,
\end{multline*}
%
where $e_x \colon \Omega_1 \to \reals$ is the indicator function of $x$. Same for the other margin.\qed

%
In conclusion, we define the \emph{transport model bundle} with margins $\gamma_1$ and $\gamma_2$ to be the sub-statistical bundle
%
\begin{equation} \label{eq:2}
  S\Gamma^\circ(\gamma_1,\gamma_2) = \setof{(\gamma,v)}{\gamma\in\Gamma^\circ(\gamma_1,\gamma_2),v \in (S_\gamma\Gamma^\circ(\mu_1,\mu_2))_{12}(\gamma)} \ .
\end{equation}

In the statistical bundle setup, at each $p \in \Delta^\circ(\Omega)$ there is a chart $s_p$ derived from the exponential representation, namely
\begin{equation*}
 s_p \colon \Delta^\circ(\Omega) \ni q \mapsto \log \frac q p - \expectat \gamma {\log \frac q p} = u \in S_p\Delta^\circ(\Omega) \ ,
\end{equation*}
so that
\begin{equation*}
  u \mapsto \euler^{u - K_p(u)} \cdot p \ , \quad K_p(u) = \log \expectat p {\euler^u} \ .
\end{equation*}
The sub-manifold of the transport model is flat in the mixture geometry and there is no simple expession of the exponential coordinate. A couple of dual parallel transports between the fibers can be computed as
\begin{gather*}
  B^*_{12}(\gamma) \ni \eta \mapsto \frac \gamma {\bar\gamma} \eta \in B^*_{1,2}(\bar\gamma) \\
  B{12}(\bar\gamma) \ni u \mapsto \Pi_{12}(\gamma) u \in B_{1,2}(\gamma) 
\end{gather*}
where $^*$ denotes the dual and $\Pi_{12}(\gamma)$ is the
$\gamma$-orthogonal projection to $B_{12}(\gamma)$. We do not further
discuss these topics here.

The simplest non-trivial case is $n_1=n_2=2$. Let
$\Omega_1=\Omega_2=\set{+1,-1}$. Any function $u$ on $\Omega$ has the
pseudo-Boolean form $u(x,y)=a_0+a_1 x + a_2 y + a_{12} xy$. In
particular a probability has the form
$\gamma(x,y) = \frac14(1+b_1x+b_2y+b_{12}xy)$ with
$\gamma_1(x) = X_{\#}\gamma (x)= \frac12(1+b_1x)$,
$\gamma_2(y) = Y_\# \gamma(y) = \frac12(1+b_2y)$. Given
$\bar b_1, \bar b_2 \in ]-1,+1[$ to fix the margins, the plan is given
by the 1 parameter family
%
  \begin{equation*}
    \gamma(x,y;\theta) = \frac14(1+\bar b_1x+\bar b_2y+\theta xy), \quad -1 + \avalof{\bar b_1+\bar b_2} < \theta < 1 - \avalof{\bar b_1 - \bar b_2} \ .
  \end{equation*}
  All the computations related to the splitting can be performed
  explicitely using the algebraic tools of polynomial algebra,
  see\cite{fontana|pistone|rogantin:2000}. In fact, the algebraic tool
  works in the general case, see \cite{pistone|rogantin:2008-jspi-1}.

  Let us discuss the optimal transport problem in the framework of the
  transport model bundle of \cref{eq:2}. Let be given a cost function
  $c \colon \Omega_1 \times \Omega_2 \to \reals$ and define the
  expected cost function
  $C \colon \Delta^\circ(\Omega)\ni \gamma \mapsto \expectat \gamma c$.

\paragraph{$\bm \models$}  \emph{The function $\gamma \mapsto C(\gamma)$ restricted to the open transport model $\Gamma^\circ(\mu_1,\mu_2)$ has statistical gradient in $S\Gamma^\circ(\mu_1,\mu_2)$ given by
%
\begin{equation}\label{eq:gradient}
 \grad C \colon \gamma \mapsto c - c_{0,\gamma} - c_{1,\gamma} - c_{2,\gamma} \in S_\gamma\Gamma^\circ(\mu_1,\mu_2) = (S_\gamma \Delta^\circ)_{12} \ . 
\end{equation}}

For each smooth curve $t \mapsto \gamma(t)$, $\gamma(0) = \gamma$, we have 
\begin{equation*}
  \derivby t C(\gamma(t)) = \derivby t \expectat {\gamma(t)} c = \expectat {\gamma(t)} {c \velocity \gamma (t)} = \scalarat {\gamma(t)} {c_{12,\gamma(t)}} {\velocity \gamma(t)} \ .
\end{equation*}
where the have used the splitting at $\gamma(t)$,
\begin{equation*}
  c = c_{0,\gamma(t)} + c_{1,\gamma(t)} + c_{2,\gamma(t)} + c_{12,\gamma(t)}
\end{equation*}
together with the fact that the velocity $\velocity\gamma(t)$ is an interaction.\qed

It follows that the equation of the \emph{gradient flow of $C$} is
%
\begin{equation*}
  \velocity \gamma = - \left(c - c_{0,\gamma} - c_{1,\gamma} - c_{2,\gamma}\right) \ .
\end{equation*}

We expect any solution $t \mapsto \gamma(t)$ of the gradient flow to converge to a coupling $\bar\gamma = \lim_{t \to \infty} \gamma(t) \in \Delta(\Omega)$ such that $\expectat {\bar\gamma} c$ is the value of the Kantorovich optimal transport problem,
%
\begin{equation*}
  \expectat {\bar\gamma} c = \min \setof{\expectat \gamma c}{ \gamma \in \Gamma(\gamma_1,\gamma_2)} \ .
\end{equation*}

Notice that the extension of the gradient in \cref{eq:gradient} to $\Delta(\Omega)$ is zero at a $\widehat\gamma$, such that
%
\begin{equation*}
  c = c_{0,\widehat \gamma} + c_{1,\widehat \gamma} + c_{2,\widehat\gamma}
\end{equation*}
%
on $\suppof{\widehat\gamma}$. This should be compared with the
properties of the optimal solutions of the Kantorovich problem, see
\cite{montrucchio|pistone:2019-arxiv:1905.07547},
\cite{pistone|rapallo|rogantin:2021}.

As a final remark, we observe that the splitting of the statistical bundle suggests that each $\gamma \in \Delta(\Omega_1\times\Omega_2)$ stays at the intersection of two orthogonal manifolds, namely, the tranport model with the margins of $\gamma$ and the additive exponential model $\euler^{u_1+u_2 - K_\gamma(u_1 + u_2)} \cdot \gamma$, $u_1 \in B_1(\gamma)$ and $u_2 \in B_2(\gamma)$.

\paragraph{Acknowledgments}
The author thanks Luigi Malag\`o and Luigi Montrucchio for their helful suggestions during the development of the present note.

% \subsection{A Subsection Sample}
% Please note that the first paragraph of a section or subsection is
% not indented. The first paragraph that follows a table, figure,
% equation etc. does not need an indent, either.

% Subsequent paragraphs, however, are indented.

% \subsubsection{Sample Heading (Third Level)} Only two levels of
% headings should be numbered. Lower level headings remain unnumbered;
% they are formatted as run-in headings.

% \paragraph{Sample Heading (Fourth Level)}
% The contribution should contain no more than four levels of
% headings. Table~\ref{tab1} gives a summary of all heading levels.

% \begin{table}
% \caption{Table captions should be placed above the
% tables.}\label{tab1}
% \begin{tabular}{|l|l|l|}
% \hline
% Heading level &  Example & Font size and style\\
% \hline
% Title (centered) &  {\Large\bfseries Lecture Notes} & 14 point, bold\\
% 1st-level heading &  {\large\bfseries 1 Introduction} & 12 point, bold\\
% 2nd-level heading & {\bfseries 2.1 Printing Area} & 10 point, bold\\
% 3rd-level heading & {\bfseries Run-in Heading in Bold.} Text follows & 10 point, bold\\
% 4th-level heading & {\itshape Lowest Level Heading.} Text follows & 10 point, italic\\
% \hline
% \end{tabular}
% \end{table}


% \noindent Displayed equations are centered and set on a separate
% line.
% \begin{equation}
% x + y = z
% \end{equation}
% Please try to avoid rasterized images for line-art diagrams and
% schemas. Whenever possible, use vector graphics instead (see
% Fig.~\ref{fig1}).

% \begin{figure}
% \includegraphics[width=\textwidth]{fig1.eps}
% \caption{A figure caption is always placed below the illustration.
% Please note that short captions are centered, while long ones are
% justified by the macro package automatically.} \label{fig1}
% \end{figure}

% \begin{theorem}
% This is a sample theorem. The run-in heading is set in bold, while
% the following text appears in italics. Definitions, lemmas,
% propositions, and corollaries are styled the same way.
% \end{theorem}
% %
% % the environments 'definition', 'lemma', 'proposition', 'corollary',
% % 'remark', and 'example' are defined in the LLNCS documentclass as well.
% %
% \begin{proof}
% Proofs, examples, and remarks have the initial word in italics,
% while the following text appears in normal font.
% \end{proof}
% For citations of references, we prefer the use of square brackets
% and consecutive numbers. Citations using labels or the author/year
% convention are also acceptable. The following bibliography provides
% a sample reference list with entries for journal
% articles~\cite{ref_article1}, an LNCS chapter~\cite{ref_lncs1}, a
% book~\cite{ref_book1}, proceedings without editors~\cite{ref_proc1},
% and a homepage~\cite{ref_url1}. Multiple citations are grouped
% \cite{ref_article1,ref_lncs1,ref_book1},
% \cite{ref_article1,ref_book1,ref_proc1,ref_url1}.
%
% ---- Bibliography ----
%
% BibTeX users should specify bibliography style 'splncs04'.
% References will then be sorted and formatted in the correct style.
%
\bibliographystyle{splncs04}
\bibliography{tutto}
%
% \begin{thebibliography}{8}
% \bibitem{ref_article1}
% Author, F.: Article title. Journal \textbf{2}(5), 99--110 (2016)

% \bibitem{ref_lncs1}
% Author, F., Author, S.: Title of a proceedings paper. In: Editor,
% F., Editor, S. (eds.) CONFERENCE 2016, LNCS, vol. 9999, pp. 1--13.
% Springer, Heidelberg (2016). \doi{10.10007/1234567890}

% \bibitem{ref_book1}
% Author, F., Author, S., Author, T.: Book title. 2nd edn. Publisher,
% Location (1999)

% \bibitem{ref_proc1}
% Author, A.-B.: Contribution title. In: 9th International Proceedings
% on Proceedings, pp. 1--2. Publisher, Location (2010)

% \bibitem{ref_url1}
% LNCS Homepage, \url{http://www.springer.com/lncs}. Last accessed 4
% Oct 2017
% \end{thebibliography}
\end{document}
